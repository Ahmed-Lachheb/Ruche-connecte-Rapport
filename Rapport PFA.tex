\documentclass[a4paper,12pt]{report}
\usepackage{graphicx}
\graphicspath{ {figures/} }
\usepackage{array}
\usepackage{lipsum}
\usepackage{float}
\usepackage{siunitx}
\usepackage[T1]{fontenc}
\usepackage[utf8]{inputenc}
\usepackage[belowskip=-10pt,aboveskip=10 pt]{caption}
\usepackage{setspace}
\doublespacing
\setlength{\intextsep}{20pt plus 2pt minus 2pt}
\setlength{\belowcaptionskip}{-15pt}
\usepackage{natbib}
\usepackage[french]{babel}
\usepackage{lastpage}
\usepackage{fancyhdr}
\lhead{ISET BEJA}
\renewcommand{\headrulewidth}{1mm}
\renewcommand{\footrulewidth}{1mm}
\begin{document}
	\pagestyle{empty}
	%	\tableofcontents
	%	\listoffigures
	%	\listoftables
	\chapter{Etat de l'art}
	\section*{Contexte }
Le contexte actuel de l'apiculture est marqué par des défis considérables auxquels les apiculteurs du monde entier sont confrontés. Les abeilles, jouant un rôle vital dans la pollinisation des cultures, sont confrontées à une multitude de menaces, notamment le changement climatique, l'utilisation intensive de pesticides, les maladies et la perte d'habitats naturels. Ces facteurs ont contribué à un déclin alarmant des populations d'abeilles, mettant en péril la sécurité alimentaire et l'équilibre écologique.

Dans ce contexte, les apiculteurs sont confrontés à la nécessité urgente de moderniser leurs pratiques de gestion des ruches pour garantir la survie des abeilles, tout en assurant une production de miel stable. L'inspection manuelle des ruches représente actuellement un défi majeur, nécessitant des déplacements fréquents dans des conditions parfois difficiles, avec des résultats limités en termes de surveillance précise.

Le besoin de surveillance en temps réel des ruches se fait sentir de plus en plus, afin de mieux comprendre les conditions environnementales spécifiques auxquelles les abeilles sont exposées. La détection précoce des anomalies, telles que les changements brusques de température ou la présence de pesticides, est cruciale pour la prévention des maladies et la protection des colonies d'abeilles.

Afin de répondre à ces enjeux, notre projet de ruche connectée vise à intégrer des technologies IoT avancées pour offrir aux apiculteurs une solution moderne, automatisée et intelligente. Cette ruche connectée permettra une surveillance constante, une collecte de données précises, des alertes en temps réel et une gestion à distance, contribuant ainsi à la préservation des abeilles et à l'optimisation de la production de miel dans un contexte apicole en évolution rapide.
	\section*{Problématique}
Les abeilles, acteurs incontournables de l'équilibre écologique, font face à une menace croissante résultant du développement humain, de l'utilisation généralisée de pesticides, des maladies et du changement climatique. Cette situation se traduit par une diminution alarmante de leurs populations, ce qui soulève des préoccupations cruciales quant à la préservation de ces pollinisateurs essentiels. Face à ces défis urgents, les apiculteurs sont contraints d'intensifier leur surveillance des ruches, non seulement pour suivre la production de miel, mais également pour détecter précocement toute anomalie potentiellement dévastatrice.

Cependant, l'inspection manuelle des ruches se heurte à des obstacles significatifs. Les déplacements fréquents nécessaires pour un contrôle efficace entraînent une perte considérable de temps et d'efforts. De plus, la mesure manuelle de paramètres météorologiques tels que la température et l'humidité est entravée par divers facteurs perturbateurs. La pesée manuelle individuelle des ruches peut causer des dommages aux colonies d'abeilles, tandis que l'exposition aux pesticides aggrave la santé des abeilles, contribuant au déclin des populations.

La gestion d'un grand nombre de ruches engendre également des pertes de temps lors de la collecte des mesures, tandis que l'organisation manuelle des informations rend difficile l'identification des zones propices à une production optimale de miel.

Dans ce contexte, notre projet ambitieux vise à concevoir et déployer une ruche connectée intelligente. Cette solution novatrice s'appuie sur des technologies de pointe pour optimiser la surveillance des ruches, réduire les contraintes opérationnelles pesant sur les apiculteurs et jouer un rôle essentiel dans la préservation des abeilles, ces précieux pollinisateurs indispensables à notre écosystème.
	
	\section*{Cahier de charge}
	Pour que le système soit efficace et bien adapté aux besoins de nos apiculteurs, il
	est nécessaire qu’il obéi à un certain nombre de critères, d’exigences et de contraintes. Ces différents points doivent être le résultat d’une réflexion profonde qui prennent en compte :
	
	
	\begin{itemize}
		\item [$\bullet$] Les besoins spécifiques de l’apiculteur
		\item [$\bullet$]Le système doit être concurrentiel
		\item [$\bullet$] La nature du lieu où le système doit être implanté 
		\item [$\bullet$]Les effets nuisibles du système sur le ruché 
		\item [$\bullet$]Les bénéfices escomptés du système 
		\item [$\bullet$] L’efficacité du système 
		\item [$\bullet$] L’estimation du prix du système 
		
	\end{itemize}
	
	\textbf{Pour la réalisation de notre système nous avons fixé un certain nombre d’exigence et de contraintes qui peuvent être énuméré comme suit }
	
	\begin{itemize}
		\item [$\bullet$] Le système doit fonctionner de manière autonome  
		\item [$\bullet$] Le système doit être reconfigurable 
		\item [$\bullet$]Le système doit être facilement intégrables dans les ruches 
		\item [$\bullet$]Transfert de données sans fil vers un stockage de données centralisé 
		\item [$\bullet$] Les données mesurées doivent également être accessibles à un apiculteur sans transfert vers une base de données centrale 
		\item [$\bullet$] Faible consommation d'énergie 
		
		\item [$\bullet$]Un coût réduit pour la réalisation du système
	\end{itemize}
	\textbf{Le système est censé mesurer les paramètres suivants :}
	\begin{itemize}
		\item  [$\bullet$] La température dans la ruche 
		\item [$\bullet$] L’humidité sous la couverture de la ruche 
		\item [$\bullet$] Le poids de la ruche 
		\item [$\bullet$] Les données climatique (Température, Humidité, …) 
		\item [$\bullet$] La concentration de monoxyde de carbone 
		\item [$\bullet$] En plus, il doit fournir des données de diagnostic.
	\end{itemize}
	
\section*{Solution proposé}	
Pour répondre aux défis mentionnés dans notre cahier des charges et créer une ruche connectée innovante, nous proposons une solution complète et intelligente, intégrant des technologies de pointe. Voici les principales caractéristiques de notre solution :

\begin{figure} [H]
	\begin{center}
		\centering
		\hspace*{-1.5cm}
		\fbox{\includegraphics[width=1.2\linewidth]{../../../Downloads/2024-03-10 (4)}}
	\end{center}
	\caption{Agile Scrum}
\end{figure}
\begin{itemize}
\item [$\bullet$] \textbf{Capteurs Intelligents :}

\textbf{Capteur de Température et d'Humidité :} Placé stratégiquement dans la ruche pour surveiller l'environnement interne.

\textbf{Capteur de Poids :} Intégré discrètement pour évaluer le poids de la ruche sans perturber les abeilles.

\textbf{Capteur de Gaz :} Détection précise du monoxyde de carbone pour assurer la sécurité des abeilles.

\item [$\bullet$] \textbf{Système de Communication IoT :}

\textbf{Communication sans Fil :} Transfert des données vers une plateforme cloud centralisée.

\item [$\bullet$] \textbf{Plateforme Cloud et Application Mobile :}

\textbf{Stockage Centralisé :} Conservation sécurisée des données pour une analyse ultérieure.

\textbf{Interface Utilisateur Conviviale :} Application mobile permettant aux apiculteurs de visualiser les données, de recevoir des alertes et de prendre des décisions informées.


\item [$\bullet$] \textbf{Coût Abordable :}

\textbf{Choix de Composants Économiques :} Sélection de composants offrant un excellent rapport qualité-prix.


\item [$\bullet$] \textbf{Installation Facile dans les Ruches :}

\textbf{Conception Ergonomique :} Intégration discrète dans la ruche avec un minimum de perturbation pour les abeilles.

\textbf{Instructions Claires :} Documentation détaillée pour faciliter l'installation par les apiculteurs.

En intégrant ces éléments, notre solution de ruche connectée ambitionne de révolutionner la gestion apicole en offrant une surveillance précise, des alertes en temps réel, une facilité d'utilisation et une autonomie énergétique. Elle vise à préserver la santé des abeilles et à optimiser la production de miel dans le contexte apicole en constante évolution.
\end{itemize}
\end{document}