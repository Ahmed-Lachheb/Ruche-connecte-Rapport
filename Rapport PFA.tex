\documentclass[a4paper,12pt]{report}
\usepackage{graphicx}
\graphicspath{ {figures/} }
\usepackage{array}
\usepackage{lipsum}
\usepackage{float}
\usepackage{siunitx}
\usepackage[T1]{fontenc}
\usepackage[utf8]{inputenc}
\usepackage[belowskip=-10pt,aboveskip=10 pt]{caption}
\usepackage{setspace}
\doublespacing
\setlength{\intextsep}{20pt plus 2pt minus 2pt}
\setlength{\belowcaptionskip}{-15pt}
\usepackage{natbib}
\usepackage[french]{babel}
\usepackage{lastpage}
\usepackage{fancyhdr}
\lhead{ISET BEJA}
\renewcommand{\headrulewidth}{1mm}
\renewcommand{\footrulewidth}{1mm}
\begin{document}
	\pagestyle{empty}
	%	\tableofcontents
	%	\listoffigures
	%	\listoftables
	\chapter{ }
	\section*{Contexte }
Dans le contexte de ce projet de ruche connectée, le besoin principal est de moderniser et d'optimiser la gestion apicole en utilisant des technologies IoT pour surmonter les défis rencontrés par les apiculteurs. Pour illustrer l'urgence de la situation, le déclin alarmant des populations d'abeilles est devenu une préoccupation mondiale, notamment en raison du développement humain, de l'utilisation de pesticides, des maladies et du changement climatique. Face à ces défis pressants, les apiculteurs se trouvent contraints de mettre en place une surveillance régulière et systématique de leurs ruches, non seulement pour suivre la production de miel, mais également pour détecter toute anomalie potentielle.

Cependant, l'inspection manuelle des ruches représente un défi majeur. Les déplacements fréquents dans les champs, nécessaires à un contrôle efficace, entraînent une perte de temps et d'efforts considérable. De plus, la précision des mesures météorologiques, telles que la température et l'humidité, est compromise en raison de divers paramètres perturbateurs. La mesure manuelle du poids individuel de chaque ruche peut causer des dommages aux colonies d'abeilles, tandis que l'exposition aux pesticides constitue une contrainte supplémentaire, affectant la santé des abeilles et contribuant au déclin de leurs populations.

Notre projet de ruche connectée vise à répondre à ces défis en proposant une solution complète qui intègre des technologies de pointe pour optimiser la surveillance des ruches, alléger les contraintes opérationnelles pesant sur les apiculteurs et jouer un rôle essentiel dans la préservation des abeilles, ces précieux pollinisateurs de notre écosystème.
	\section*{Problématique}
	Les abeilles, jouant un rôle essentiel dans l'équilibre écologique, font actuellement face à des menaces sérieuses, notamment le développement humain, l'utilisation de pesticides, les maladies et le changement climatique, entraînant une diminution alarmante de leurs populations. Face à ces défis pressants, les apiculteurs se trouvent contraints de mettre en place une surveillance régulière et systématique de leurs ruches, non seulement pour suivre la production de miel, mais également pour détecter toute anomalie potentielle.
	
	Cependant, l'inspection manuelle des ruches représente un défi majeur. Les déplacements fréquents dans les champs, nécessaires à un contrôle efficace, entraînent une perte de temps et d'efforts considérable. De plus, la précision des mesures météorologiques, telles que la température et l'humidité, est compromise en raison de divers paramètres perturbateurs. La mesure manuelle du poids individuel de chaque ruche peut causer des dommages aux colonies d'abeilles, tandis que l'exposition aux pesticides constitue une contrainte supplémentaire, affectant la santé des abeilles et contribuant au déclin de leurs populations.
	
	La gestion d'un grand nombre de ruches entraîne également une perte significative de temps lors de la collecte des mesures. De plus, l'organisation manuelle des informations provenant des ruches rend les études visant à identifier les zones propices à une production optimale de miel difficiles et imprécises.
	
	Notre ambitieux projet aspire à concevoir et déployer une ruche connectée intelligente, représentant une solution novatrice destinée à révolutionner la gestion des colonies d'abeilles. Confrontés aux défis complexes actuels de l'apiculture, notre ruche intelligente s'engage à intégrer des technologies de pointe afin d'optimiser la surveillance des ruches, alléger les contraintes opérationnelles pesant sur les apiculteurs, et jouer un rôle essentiel dans la préservation des abeilles, ces précieux pollinisateurs de notre écosystème.
	
	
	\section*{Cahier de charge}
	Pour que le système soit efficace et bien adapté aux besoins de nos apiculteurs, il
	est nécessaire qu’il obéi à un certain nombre de critères, d’exigences et de contraintes. Ces différents points doivent être le résultat d’une réflexion profonde qui prennent en compte :
	
	
	\begin{itemize}
		\item [$\bullet$] Les besoins spécifiques de l’apiculteur
		\item [$\bullet$]Le système doit être concurrentiel
		\item [$\bullet$] La nature du lieu où le système doit être implanté 
		\item [$\bullet$]Les effets nuisibles du système sur le ruché 
		\item [$\bullet$]Les bénéfices escomptés du système 
		\item [$\bullet$] L’efficacité du système 
		\item [$\bullet$] L’estimation du prix du système 
		
	\end{itemize}
	
	\textbf{Pour la réalisation de notre système nous avons fixé un certain nombre d’exigence et de contraintes qui peuvent être énuméré comme suit }
	
	\begin{itemize}
		\item [$\bullet$] Le système doit fonctionner de manière autonome  
		\item [$\bullet$] Le système doit être reconfigurable 
		\item [$\bullet$]Le système doit être facilement intégrables dans les ruches 
		\item [$\bullet$]Transfert de données sans fil vers un stockage de données centralisé 
		\item [$\bullet$] Les données mesurées doivent également être accessibles à un apiculteur sans transfert vers une base de données centrale 
		\item [$\bullet$] Faible consommation d'énergie 
		
		\item [$\bullet$]Un coût réduit pour la réalisation du système
	\end{itemize}
	\textbf{Le système est censé mesurer les paramètres suivants :}
	\begin{itemize}
		\item  [$\bullet$] La température dans la ruche 
		\item [$\bullet$] L’humidité sous la couverture de la ruche 
		\item [$\bullet$] Le poids de la ruche 
		\item [$\bullet$] Les données climatique (Température, Humidité, …) 
		\item [$\bullet$] La concentration de monoxyde de carbone 
		\item [$\bullet$] En plus, il doit fournir des données de diagnostic.
	\end{itemize}
	
	
	
	
	
	
	
	
	
	
	
	
	
	
	
	
	
\end{document}